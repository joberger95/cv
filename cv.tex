\documentclass[10pt,a4paper]{moderncv}
\moderncvtheme[green]{classic}
\usepackage{hyperref}
\usepackage[utf8]{inputenc}%
\usepackage[latin1]{inputenc}%
\usepackage{lmodern}%
\usepackage[scale=0.85]{geometry} % La taille pris par le contenu, ici on a 15% de marges.
\nopagenumbers{} % Permet de masquer les numéros de page
\moderncvicons{awesome}

\hypersetup{
   pdfauthor   = {Moi},
   pdftitle    = {CV},
   colorlinks=true,
   breaklinks=true,
   urlcolor= blue,
   linkcolor= blue,
   pdfcreator  = {\LaTeX},
   pdfproducer = {Kile}
}

\title{Etudiant ingénieur, Mastère Expert Mécatronique }
\firstname{Jordan}
\familyname{BERGER}
\address{14 rue Louis Liard}{33000 Bordeaux}
\mobile{06 48 20 09 53}
\email{jordan.berger@ynov.com}
\social[linkedin][www.linkedin.com/in/jordan-berger-1b1814162]{Jordan BERGER}
\social[github]{joberger95}
\extrainfo{\\}
\extrainfo{25 ans\\Permis A,B}
 
\begin{document}
\maketitle{}

\section{Expériences professionnelles}
\cventry{2016--2019}{Equipier polyvalent}{McDonald's}{Saran-Loiret, Rouen--Seine-Maritime}%
            {}{}

\cventry{2014--2015}{Elève Sous-Officier, spécialité CNS-Communication Navigation Surveillance}{BA721}{Rochefort--Charente-Martime}%
            {}{}

\cventry{2013--2016}{Missions intérim}{}{}%
            {}{}

\cventry{Juin 2017}{Stage BTS}{IPROCIA}{Orléans-Loiret}%
            {Développement d'un programme de géolocalisation en temps réel sur Arduino,\href{https://www.linkedin.com/in/jordan-berger-1b1814162/}{Lien}}%
            {}{}
 
\section{Formation}
\cventry{Sept 2019--Maintenant}{Etudiant ingénieur, Mastère Expert Mécatronique, robotique et ingénierie système}{Ynov Campus}{Bordeaux--Gironde}%
            {}{}
\cventry{Sept 2018--Mai2019}{Etudiant ingénieur Electronique Générale}{ESIGELEC}{Saint-Etienne-Du-Rouvray--Seine-Maritime}%
            {}{}
\cventry{Sept 2016--Juin 2018}{Diplôme de technicien supérieur Systèmes Electronique Numérique, option électronique et communication}{Lycée Maurice Genevoix}{Ingré--Loiret}%
            {}%
            {}

\section{Compétences}

\cvitem{Langages}{C/C++,Python,$\mu$Python,HTML5/CSS3}
\cvitem{Outils}{GitHub}
\cvitem{Linux}{
\begin{itemize}%
\item Scripts 
\item Chaîne de compilation croisée
\item Outils: crosstool-ng, U-Boot, Noyau Linux
\end{itemize}}
\cvitem{Electronique}{
\begin{itemize}%
\item Mise en oeuvre de systèmes
\item Routage/Soudure
\item Analogique:transistors, diodes, AOP
\item Communication: UART, I$^2$C, GSM, GPS-NMEA0183
\item Microcontrôleur: série PIC 18F, ARM Cortex M-4
\item Servo-moteur 
\end{itemize}}
 
\section{Langues}
\cvline{Français}{Langue maternelle}
\cvline{Anglais}{TOEIC: 520 en 2019}

\section{Projets}

\cventry{}{Projet étude}{Hexapode, premières notions de la robotique, mise en oeuvre}{}%
        {\href{https://github.com/joberger95/Hexapode}{Lien}}{}
\cventry{}{Projet étude}{Space Invaders, jeux-vidéo embarqué, STM32F4 discovery}{}%
        {\href{https://github.com/joberger95/space_invaders}{Lien}}{}   
\cventry{}{Projet étude}{Carte Numérique, centralisation de données, mise en oeuvre  de fonctions}{}%
        {\href{https://github.com/joberger95/projet_BTS}{Lien}}{}  
\cventry{}{Projet perso'}{OBD,apprentissage et mise en oeuvre de notions Python}{}%
        {\href{https://github.com/joberger95/OBD}{Lien}}{} 

\fancyfoot[L]{Ecrit en LaTeX}
\end{document}
